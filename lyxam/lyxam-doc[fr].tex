\documentclass[12pt,a4paper]{scrartcl}

\usepackage[utf8]{inputenc}
\usepackage[T1]{fontenc}
\usepackage{ucs}

\usepackage[french]{babel,varioref}

\usepackage{multicol}
\usepackage{subfig}
\usepackage{enumitem}

\usepackage{color}
\usepackage{hyperref}
\hypersetup{
    colorlinks,
    citecolor=black,
    filecolor=black,
    linkcolor=black,
    urlcolor=black
}

\usepackage{amsthm}

\usepackage{tcolorbox}
\tcbuselibrary{listingsutf8}

\usepackage{ifplatform}

\usepackage{xstring}

\usepackage{fancyvrb}


% MISC

\tcbset{%
	sharp corners,%
	left=1mm, right=1mm,%
	bottom=1mm, top=1mm,%
	colupper=red!75!blue%
}

\setlength{\parindent}{0cm}

\theoremstyle{definition}
\newtheorem*{remark}{Remarque}

\usepackage[raggedright]{titlesec}

\titleformat{\paragraph}[hang]{\normalfont\normalsize\bfseries}{\theparagraph}{1em}{}
\titlespacing*{\paragraph}{0pt}{3.25ex plus 1ex minus .2ex}{0.5em}


\makeatletter
    \newcommand\resetallcnt{
    	\setcounter{lyxam@counter@topic}{0}
    	\setcounter{lyxam@counter@exercise}{0}
    	\setcounter{lyxam@counter@problem}{0}
    	\setcounter{lyxam@counter@bonus}{0}
    	\setcounter{lyxam@counter@subpart}{0}
    }
\makeatother

% Technical IDs

\newwrite\tempfile

\immediate\openout\tempfile=x-\jobname.macros-x.txt

\AtEndDocument{\immediate\closeout\tempfile}


\newcommand\IDconstant[1]{%
    \immediate\write\tempfile{constant@#1}%
}


\makeatletter
\newcommand\IDmacro{\@ifstar{\@IDmacroStar}{\@IDmacroNoStar}}

\newcommand\@IDmacroNoStar[3]{%
    \texttt{%
    	\textbackslash#1%
    	\IfStrEq{#2}{0}{}{%
    		\,\,[#2 Option%
			\IfStrEq{#2}{1}{}{s}]%
		}%
	    \IfStrEq{#3}{}{}{%
    		\,\,(#3 Argument%
			\IfStrEq{#3}{1}{}{s})%
		}
   	}
    \immediate\write\tempfile{macro@#1@#2@#3}%
}

\newcommand\@IDmacroStar[2]{%
    \@IDmacroNoStar{#1}{0}{#2}%
}


\newcommand\IDenv{\@ifstar{\@IDenvStar}{\@IDenvNoStar}}

\newcommand\@IDenvNoStar[3]{%
    \texttt{%
    	\textbackslash#1%
    	\IfStrEq{#2}{0}{}{%
    		\,\,[#2 Option%
			\IfStrEq{#2}{1}{}{s}]%
		}%
	    \IfStrEq{#3}{}{}{%
    		\,\,(#3 Argument%
			\IfStrEq{#3}{1}{}{s})%
		}
   	}
    \immediate\write\tempfile{env@#1@#2@#3}%
}

\newcommand\@IDenvStar[2]{%
    \@IDenvNoStar{#1}{0}{#2}%
}


\newcommand\@IDoptarg{\@ifstar{\@IDoptargStar}{\@IDoptargNoStar}}

\newcommand\@IDoptargStar[2]{%
	\vspace{0.5em}
	--- \texttt{#1%
		\IfStrEq{#2}{}{:}{\,#2:}%
	}%
}

\newcommand\@IDoptargNoStar[2]{%
	\IfStrEq{#2}{}{%
		\@IDoptargStar{#1}{}%
	}{%
		\@IDoptargStar{#1}{\##2}%
	}%
}


\newcommand\IDkey[1]{%
	\@IDoptarg*{Option}{{\itshape "#1"}}%
}


\newcommand\IDoption[1]{%
	\@IDoptarg{Option}{#1}%
}


\newcommand\IDarg[1]{%
	\@IDoptarg{Argument}{#1}%
}
\makeatother

\usepackage[apmep, fr]{lyxam}


\begin{document}

\title{%
	Le package \texttt{lyxam}:\\%
	des mises en forme clés en main\\%
	pour des fiches d'exercices\\%
	{\footnotesize Code source disponible sur \url{https://github.com/bc-latex/ly-xam}.}\\%
	{\footnotesize Version \texttt{0.1.0-beta} développée et testée sur \macosxname{}.}%
}
\author{Christophe BAL}
\date{2017-11-12}

\maketitle


\vspace{2em}

\hrule

\tableofcontents

\vspace{1.5em}

\hrule

\newpage



\section{Introduction}

Le but du tout petit package \verb+lyxam+ est de fournir un moyen simple de rédiger des feuilles d'exercices pour des entraînements ou des évaluations dans le cadre d'un enseignement.
%Pour le moment, un seul type de mise en forme est proposé mais il est prévu d'en proposer plus à l'avenir 
%\footnote{
%	De plus, le code \LaTeX{} va évoluer pour faciliter l'ajout de nouvelles mises en forme par des personnes connaissant un peu la programmation \LaTeX{}.
%}.




\section{Les options du package}

Avant d'entrer dans le vif du sujet, nous donnons ici toutes les options utilisables lors de l'appel du package via \verb+\usepackage[...]{lyxam}+.

\begin{itemize}[label=\textbullet]
%	\item \verb+apmep+ indique d'utiliser le style de mise en forme nommé "apmep" en référence aux sujets de BAC proposés sur le site de \href{https://www.apmep.fr}{l'APMEP}, l'Association des Professeurs de Mathématiques de l'Enseignement public. C'est le seul type de mise en forme disponible pour le moment !

	\item \verb+en+, valeur par défaut
	\footnote{
		Sorry for the french frogs that we are...
	}, et \verb+fr+ permettent d'indiquer d'utiliser l'anglais ou le français pour tous les textes ajoutés par \verb+lyxam+.

	\item \verb+hf+, valeur par défaut, et \verb+nohf+ demandent d'afficher ou non les en-têtes et les pieds de page.
	Les lettres \verb+hf+ sont pour "\textbf{h}-eaders" et "\textbf{f}-ooters", soit "en-tête" et "pied de page" en français.

	\item \verb+pts+, valeur par défaut, et \verb+nopts+ servent à voir ou non les points donnés pour chaque exercice.

	\item \verb+noshort+, valeur par défaut, et \verb+short+ demandent d'utiliser le nom long ou abrégé des types d'exercices.

	\item \verb+src+, valeur par défaut, et \verb+nosrc+ feront afficher ou non les sources utilisées.
\end{itemize}

%\begin{remark}
%	Dans la suite de cette documentation, nous utilisons les options \verb+apmep+ et \verb+fr+, c'est à dire via \verb+\usepackage[apmep, fr]{lyxam}+.
%\end{remark}
\begin{remark}
	Dans la suite de cette documentation, nous invoquons juste l'option \verb+fr+.Autrement dit, nous passons par \verb+\usepackage[fr]{lyxam}+.
\end{remark}




\section{Quel devoir donnez-vous ?}

	\subsection{La commande \texttt{\textbackslash exam}}

La commande \verb+\exam+ permet de donner des informations générales à propos du devoir. 
Le code ci-dessous utilise toutes les options disponibles, et la figure \ref{style:apmep} \vpageref{style:apmep} donne un aperçu du rendu obtenu dans ce cas. 

\begin{tcblisting}{listing only}
\exam[deliver,%
      kind     = D.S.,%
      nb       = 1,%
      subnb    = Sujet A,%
      subject  = Mathématiques,%
      theme    = Probabilités,%
      sector   = Série Scientifique,%
      class    = 1S4,%
      location = Lycée MONGE (Chambéry),%
      date     = 20/10/2017,%
      time     = 2h]
\end{tcblisting}


\begin{figure}[!tbp]
  \setlength{\fboxrule}{1.5pt}
  \centering
  \fbox{\includegraphics[width=0.47\linewidth]{example-doc[fr]-0.jpg}}
  \hfill
  \fbox{\includegraphics[width=0.47\linewidth]{example-doc[fr]-1.jpg}}
  \caption{Style \texttt{apmep}.}
  \label{style:apmep}
\end{figure}

Expliquons maintenant en détail le rôle de chacun des paramètres qui sont tous optionnels. Lorsqu'aucune valeur par défaut n'est indiquée, c'est que cette valeur est un texte vide.

\begin{enumerate}
	\item \verb+deliver+, valant \verb+false+ par défaut, est pour un sujet à rendre avec la copie. 
	\begin{itemize}[label=\textbullet]
		\setlength\itemsep{0em}

		\item On peut juste indiquer \verb+deliver+ au lieu de \verb+deliver = true+ où \emph{"true"} signifie \emph{"vrai"}.

		\item Ne pas utiliser l'option revient à passer par \verb+deliver = false+ où \emph{"false"} signifie \emph{"faux"}.
	\end{itemize}
	Ceci a pour conséquence que le sujet débutera, ou non, par une zone où l'élève devra indiquer son nom et son prénom.

	\item \verb+kind+, valant \verb+Test+ par défaut, est le type de devoir : un \emph{"D.S."}, un \emph{"D.M."}, une \emph{"Interrogation Surprise"}, une \emph{"Fiche d'entraînement"}, une \emph{"Activité"}...
	Vous noterez que le terme \emph{"devoir"} ne se limite pas juste aux devoirs notés.

	\item \verb+nb+ est le numéro du devoir.

	\item \verb+subnb+ permet d'indiquer une sorte de numérotation secondaire. C'est utile par exemple pour indiquer \emph{"Sujet A"}, \emph{"Sujet B"} ...

	\item \verb+subject+ permet si besoin de donner la thématique générale du devoir comme par exemple \emph{"Mathématiques"}, \emph{"Informatique Générale"}...

	\item \verb+theme+ sert à compléter la thématique générale en indiquant un ou des points particuliers comme par exemple \emph{"Probabilités"}, \emph{"Réseaux"} ...

	\item \verb+sector+ sert à indiquer une section, au sens administratif, à laquelle s'adresse le devoir. Par exemple, pour un sujet de Baccalauréat S en France, on serait amené à utiliser \verb+sector = Série Scientifique+.

	\item \verb+class+ indique la classe et/ou le groupe auquel est destiné le devoir.

	\item \verb+location+ vous permet d'indiquer un lieu géographique, typiquement un établissement scolaire ou universitaire.

	\item \verb+date+ est pour la date du devoir.

	\item \verb+time+ est pour la durée du devoir.
\end{enumerate}


%	\subsection{Différentes mises en forme}
%
%Le package est fourni avec des exemples de fichiers \LaTeX{} afin d'observer ce que permet \verb+lyxam+. Explorez-les !


	\subsection{Fiche technique}

\IDmacro{exam}{11}{}

\IDkey{kind} le type de devoir, la valeur par défaut étant \verb+Test+.

\IDkey{render} pour un sujet à rendre, ou non, avec une zone pour le nom et le prénom de l'étudiant. Deux valeurs possibles : \verb+false+, valeur par défaut, ou \verb+true+.

\IDkey{nb} le numéro du devoir.

\IDkey{subnb} une numérotation secondaire du devoir.

\IDkey{subject} la matière, le sujet du devoir.

\IDkey{theme} un sous-thème ou une sous-partie de la matière ou du sujet du devoir.

\IDkey{sector} un secteur, une section, au sens administratif, à laquelle s'adresse le devoir.

\IDkey{class} la classe concernée par le devoir.

\IDkey{location} le lieu géographique où a lieu le devoir.

\IDkey{date} le texte donnant la date du devoir.

\IDkey{time} le texte indiquant juste la durée du devoir.




\newcommand\exosoptions{
\IDkey{pts} le nombre de points avec le cas particulier de $0$ qui demande d'afficher "Non noté".

\IDkey{time} la durée de l'exercice.

\IDkey{id} un texte de votre choix pour remplacer le numéro (ceci a pour effet de bloquer temporairement la numérotation).

\IDkey{title} un titre.

\IDkey{about} une petite indication liée à l'exercice (comme par exemple qu'il ne s'adresse qu'aux élèves motivés).

\IDkey{src} la source utilisée pour confectionner l'exercice.
}


\section{Indiquer vos exercices}

    \subsection{Important pour la suite}

%Rappelons que tous les exemples utilisés dans cette documentation ont été obtenus en utilisant \verb+\usepackage[apmep, fr]{lyxam}+ dans le préambule.

Rappelons que tous les exemples utilisés dans cette documentation ont été obtenus en utilisant \verb+\usepackage[fr]{lyxam}+ dans le préambule.


    \subsection{Différents types d'exercices}

Voici l'ensemble des commandes disponibles pour indiquer un type d'exercice.

% All kind of level 2 contexts - START
\begin{itemize}[label=\textbullet]
\makeatletter
    \item \verb+\activity+ correspond à "\lyxam@text@activity{}".
    
    \item \verb+\bonus+ correspond à "\lyxam@text@bonus{}".
    
    \item \verb+\exercise+ correspond à "\lyxam@text@exercise{}".
    
    \item \verb+\mcq+ correspond à "\lyxam@text@mcq{}".
    
    \item \verb+\praticalwork+ correspond à "\lyxam@text@praticalwork{}".
    
    \item \verb+\problem+ correspond à "\lyxam@text@problem{}".
\makeatother
\end{itemize}
% All kind of level 2 contexts - END

Dans la suite, nous donnerons des exemples principalement avec la commande \verb+\exercise+. Ceci n'est pas gênant car le principe reste identique pour les autres commandes.


    \subsection{Numérotation minimaliste}

En \textbf{compilant deux fois} au moins, \verb+lyxam+ va pouvoir donner une numérotation minimaliste de vos exercices. Tout d'abord, ci-dessous on obtient classiquement des exercices tous numérotés où vous noterez que chaque type d'exercice admet sa propre numérotation.


\begin{tcblisting}{}
\exercise
Bla, bla, bla, bla, bla, ...

\exercise
Bla, bla, bla, bla, bla, ...

\problem
Bla, bla, bla, bla, bla, ...

\problem
Bla, bla, bla, bla, bla, ...
\end{tcblisting}


Que souhaite-t-on avoir dans un sujet ne contenant qu'un seul problème et un seul bonus ? Nul besoin ici d'avoir un numéro pour ces derniers. C'est ce que fait \verb+lyxam+.

\resetallcnt{}

\begin{tcblisting}{}
\exercise
Bla, bla, bla, bla, bla, ...

\exercise
Bla, bla, bla, bla, bla, ...

\problem
Bla, bla, bla, bla, bla, ...

\bonus
Bla, bla, bla, bla, bla, ...
\end{tcblisting}



    \subsection{Indiquer le barème}

La commande \verb+\exercise+ dispose de plusieurs options dont l'une permet de donner le nombre de points attribués à un exercice où lorsque l'on indique $0$ point le package indique une exercice non noté (voir la section suivante pour un exemple). Voici comment donner un barème.

\resetallcnt{}

\begin{tcblisting}{}
\exercise[pts = 5]
Bla, bla, bla, bla, bla, ...
\end{tcblisting}


    \subsection{Toutes les options en action}

Les fiches techniques données un peu plus bas expliquent le champs d'utilisation de chaque option.

\begin{tcblisting}{}
\exercise[pts   = 0,
          time  = 3 jours,
          id    = facultatif,
          title = Devinette,
          about = Pour spécialistes uniquement,
          src   = Le livre des experts]
Bla, bla, bla, bla, bla, ...
\end{tcblisting}



    \subsection{Cacher le texte "Exercice"}

La version étoilée \verb+\exercise*+ permet de cacher le contexte qui du point de vue de \verb+lyxam+ sera ici le texte "Exercice". Ceci est surtout pratique pour les sous-parties présentées un peu plus bas. Dans ce genre de situation, il faut obligatoirement donner un titre.

\begin{tcblisting}{}
\exercise*[title = Juste mon titre]

Bla, bla, bla, bla, bla, ...
\end{tcblisting}

    \subsection{Fiches techniques}

Toutes les options données ci-dessous sont facultatives.

\bigskip


% IDmacro - All kind of level 2 contexts - START
\IDmacro{activity}{6}{}

\IDmacro{bonus}{6}{}

\IDmacro{exercise}{6}{}

\IDmacro{mcq}{6}{}

\IDmacro{praticalwork}{6}{}

\IDmacro{problem}{6}{}
% IDmacro - All kind of level 2 contexts - END

\exosoptions{}



\section{Indiquer des sous-parties dans vos exercices}

    \subsection{Un exemple suffit\dots{} ou presque}

La commande \verb+\subpart+, avec les mêmes options que les commandes de type \verb+\exercice+, sert pour des sous-parties d'un exercice qui seront numérotées relativement aux exercices comme le montre l'exemple suivant.

\resetallcnt{}

\begin{tcblisting}{}
\exercise
\subpart
\subpart

\exercise
\subpart
\subpart
\end{tcblisting}


    \subsection{Fiche technique}

Toutes les options données ci-dessous sont facultatives et identiques à celles vues précédemment.

\bigskip


% IDmacro - All kind of level 3 contexts - START
\IDmacro{subpart}{6}{}
% IDmacro - All kind of level 3 contexts - END

\exosoptions{}



\section{Regrouper vos exercices par thèmes}

    \subsection{Un exemple suffit\dots{} ou presque}

La commande \verb+\topic+, avec les mêmes options que les commandes de type \verb+\exercice+, sert juste à regrouper des exercices par thématiques : fût un temps où dans les brevets du collège, il y avait une partie numérique et une autre géométrique.
Chaque nouvelle utilisation de \verb+\topic+ remet juste à zéro la numérotation des sous-parties mais pas celles des contextes de type \verb+\exercice+.

\resetallcnt{}

\begin{tcblisting}{}
\topic
\exercise
\exercise

\topic
\exercise
\end{tcblisting}


    \subsection{Fiche technique}

Toutes les options données ci-dessous sont facultatives et identiques à celles vues précédemment.

\bigskip


% IDmacro - All kind of level 1 contexts - START
\IDmacro{topic}{6}{}
% IDmacro - All kind of level 1 contexts - END

\exosoptions{}



\section{Personnaliser les numérotations}

    \subsection{Un petit exemple}

En mettant un peu les mains dans le cambouis, on peut modifier à sa guise la numérotation de chaque type d'exercice. Voici un exemple tapé directement dans un fichier \verb+tex+, ce qui impose d'employer \verb+\makeatletter...\makeatother+.

\resetallcnt{}

\begin{tcblisting}{}
\makeatletter
    \renewcommand\lyxam@counter@exercise@style[1]{--\arabic{#1}--}
\makeatother

\exercise
\exercise
\end{tcblisting}


    \subsection{Fiches techniques}

% IDmacro - All styles for counters - START
\IDmacro{lyxam@counter@activity@style}{0}{1}

\IDmacro{lyxam@counter@bonus@style}{0}{1}

\IDmacro{lyxam@counter@exercise@style}{0}{1}

\IDmacro{lyxam@counter@mcq@style}{0}{1}

\IDmacro{lyxam@counter@praticalwork@style}{0}{1}

\IDmacro{lyxam@counter@problem@style}{0}{1}

\IDmacro{lyxam@counter@subpart@style}{0}{1}

\IDmacro{lyxam@counter@topic@style}{0}{1}
% IDmacro - All styles for counters - END

\IDarg{} the \LaTeX{} counter associated to the context.


\section{Utiliser un préambule}

    \subsection{Un exemple avec toutes les options}

L'environnement \verb+\preamble+ sert à rédiger un ou plusieurs paragraphes en préambule de tout un examen, d'un thème, d'un exercice ou d'une partie.
Il propose deux options : l'une \verb+center+ pour centrer le contenu à l'intérieur du préambule, l'autre \verb+scale = ...+ un coefficient multiplicatif pour la largeur allouée au préambule (ce coefficient est multiplié à \verb+\linewidth+ pour calculer la largeur souhaitée).

\resetallcnt{}

\begin{tcblisting}{}
\begin{preamble}[scale = 0.75, center]
    Cet exercice est indépendant \verb+;-)+. Bla, bla, bla, bla, bla, bla,
    bla, bla, bla, bla, bla, bla, bla, bla, bla, bla, bla, bla, bla,\dots
\end{preamble}

\exercise

\begin{preamble}
    Dans cet exercice, tout trace de recherche sera prise en compte à
    condition de ne pas explorer le Pôle Nord quand on s'intéresse
    au Pôle Sud. Quoique...
\end{preamble}

Trouver ....
\end{tcblisting}


    \subsection{Fiches techniques}

\IDenv{preamble}{2}{}

\IDkey{scale} un nombre, valant $1$ par défaut, qui sera multiplié à la largeur d'une ligne pour obtenir la largeur souhaitée pour le préambule.

\IDkey{center} un booléen, valant \verb+false+ par défaut, pour centrer ou non le contenu du préambule (notez que \verb+center+ est un raccourci de \verb+center = true+).





\section{Historique}

Tous les changements sont décrits en anglais uniquement dans le dossier \verb+change_log+ : voir le code source de \verb+lyxam+ sur \verb+github+. Nous ne donnons ici qu'un très bref historique de \verb+lyxam+.

\begin{description}[leftmargin=1em]
	\setlength\itemsep{1em}

	\item[À SUIVRE !] Nouvelle version à venir...
	\begin{itemize}
        \item Les nouvelles options \verb+short+ et \verb+noshort+ permettent au chargement du package de demander d'écrire les versions courtes ou longues des noms des types d'exercices.

        \item L'option \verb+render+ de la macro \verb+\exam+ a été renommée \verb+deliver+ (ce qui semble plus correct).
        
        \item Ajout de deux options à l'environnement \verb+\preamble+ : l'une pour centrer ou non le contenu, et l'autre pour choisir la largeur occupée par le préambule.

        \item Il est maintenant possible de personnaliser la numérotation des exercices.


        \item \dots
	\end{itemize}

	\item[2017-11-12] Nouvelle version mineure \verb+0.1.0-beta+ du package.
	\begin{itemize}
        \item Les nouvelles options \verb+hf+ et \verb+nohf+ permettent au chargement du package de montrer ou cacher les en-têtes et les pieds de page.

        \item L'option \verb+preamble+ de la macro \verb+\exam+ disparait pour laisser place au nouvel environnement \verb+\begin{preamble}...\end{preamble}+ utilisable n'importe où.

        \item Pour la macro \verb+\exam+, tous les paramètres sont optionnels.

        \item Pour les macros du type \verb+\exercise+, le paramètre \verb+note+ a été renommé \verb+about+.

		\item En interne, l'utilisation de \verb+simplekv+ a permis une refonte complète du code en simplifiant la méthode à utiliser pour créer de nouvelles mises en page "maison".
	\end{itemize}

	\item[2017-11-03] Première version publique \verb+0.0.0-beta+ du package.
\end{description}



\end{document}
