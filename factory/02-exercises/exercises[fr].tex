\documentclass[12pt,a4paper]{article}

\usepackage[utf8]{inputenc}
\usepackage{ucs}
\usepackage[top=2cm, bottom=2cm, left=1.5cm, right=1.5cm]{geometry}


\usepackage[apmep, fr]{exercises}

\usepackage{color}
\usepackage{hyperref}
\hypersetup{
    colorlinks,
    citecolor=black,
    filecolor=black,
    linkcolor=black,
    urlcolor=black
}

\usepackage[francais]{babel}

\usepackage{enumitem}

\usepackage{amsthm}

\usepackage{tcolorbox}
\tcbuselibrary{listingsutf8}

\usepackage{pgffor}
\usepackage{xstring}


% MISC

\setlength{\parindent}{0cm}

\usepackage[raggedright]{titlesec}

\titleformat{\paragraph}[hang]{\normalfont\normalsize\bfseries}{\theparagraph}{1em}{}
\titlespacing*{\paragraph}{0pt}{3.25ex plus 1ex minus .2ex}{0.5em}

\makeatother
	\newcommand\resetallcnt{
		\setcounter{lyxam@topic@counter}{0}
		\setcounter{lyxam@exercise@counter}{0}
		\setcounter{lyxam@problem@counter}{0}
		\setcounter{lyxam@bonus@counter}{0}
		\setcounter{lyxam@subpart@counter}{0}
	}
\makeatletter

% Technical IDs

\newwrite\tempfile

\immediate\openout\tempfile=x-\jobname.macros-x.txt

\AtEndDocument{\immediate\closeout\tempfile}

\newcommand\IDconstant[1]{%
    \immediate\write\tempfile{constant@#1}%
}

\makeatletter
    \newcommand\IDmacro{\@ifstar{\@IDmacroStar}{\@IDmacroNoStar}}
    
    \newcommand\@IDmacroNoStar[3]{%
        \texttt{%
            \textbackslash#1%
            \IfStrEq{#2}{0}{}{%
                \,\,[#2 Option%
                \IfStrEq{#2}{1}{}{s}]%
            }%
            \IfStrEq{#3}{0}{}{%
                \,\,(#3 Argument%
                \IfStrEq{#3}{1}{}{s})%
            }
           }
        \immediate\write\tempfile{macro@#1@#2@#3}%
    }

    \newcommand\@IDmacroStar[2]{%
        \@IDmacroNoStar{#1}{0}{#2}%
    }

    \newcommand\@IDoptarg{\@ifstar{\@IDoptargStar}{\@IDoptargNoStar}}
    
    \newcommand\@IDoptargStar[2]{%
        \vspace{0.5em}
        --- \texttt{#1%
            \IfStrEq{#2}{}{:}{\,#2:}%
        }%
    }

    \newcommand\@IDoptargNoStar[2]{%
        \IfStrEq{#2}{}{%
            \@IDoptargStar{#1}{}%
        }{%
            \@IDoptargStar{#1}{\##2}%
        }%
    }


    \newcommand\IDoption{\@ifstar{\@IDoptionStar}{\@IDoptionNoStar}}
    
    \newcommand\@IDoptionStar[1]{%
        \@IDoptarg*{Option}{{\itshape "#1"}}%
    }

    \newcommand\@IDoptionNoStar[1]{%
        \@IDoptarg{Option}{#1}%
    }


    \newcommand\IDarg{\@ifstar{\@IDargStar}{\@IDargNoStar}}
    
    \newcommand\@IDargStar[1]{%
        \@IDoptarg*{Argument}{{\itshape "#1"}}%
    }

    \newcommand\@IDargNoStar[1]{%
        \@IDoptarg{Argument}{#1}%
    }
\makeatother


\begin{document}

\newcommand\exosoptions{
\IDoption*{pts} le nombre de points avec le cas particulier de $0$ qui demande d'afficher "Non noté".

\IDoption*{time} la durée de l'exercice.

\IDoption*{id} un texte de votre choix pour remplacer le numéro (ceci a pour effet de bloquer temporairement la numérotation).

\IDoption*{title} un titre. 

\IDoption*{note} une petite indication liée à l'exercice (comme par exemple qu'il ne s'adresse qu'aux élèves motivés).

\IDoption*{src} la source utilisée pour confectionner l'exercice.
}


\section{Indiquer vos exercices}

    \subsection{Important pour la suite}

Tous les exemples utilisés ci-dessous ont été obtenus en indiquant dans le préambule les options suivantes : \verb+\usepackage[apmep, fr]{exercises}+.


    \subsection{Différents types d'exercices}

Voici l'ensemble des commandes disponibles pour indiquer un type d'exercice.

% All kind of level 2 contexts - START
\begin{itemize}[label=\textbullet]
\makeatletter
    \item \verb+\activity+ correspond à "lyxam@activity@text{}".
    
    \item \verb+\bonus+ correspond à "lyxam@bonus@text{}".
    
    \item \verb+\exercise+ correspond à "lyxam@exercise@text{}".
    
    \item \verb+\mcq+ correspond à "lyxam@mcq@text{}".
    
    \item \verb+\praticalwork+ correspond à "lyxam@praticalwork@text{}".
    
    \item \verb+\problem+ correspond à "lyxam@problem@text{}".
\makeatother
\end{itemize}
% All kind of level 2 contexts - END

Dans la suite, nous donnerons des exemples principalement avec la commande \verb+\exercise+. Ceci n'est pas gênant car le principe reste identique pour les autres commandes.


    \subsection{Numérotation automatique "intelligente"}

En \textbf{compilant deux fois} au moins, \verb+lyxam+ va pouvoir donner une numérotation "intelligente" de vos exercices. Ci-dessous, on obtient classiquement des exercices tous numérotés où vous noterez que chaque type d'exercice a sa propre numérotation.


\begin{tcblisting}{}
\exercise
Bla, bla, bla, bla, bla, ...

\exercise
Bla, bla, bla, bla, bla, ...

\problem
Bla, bla, bla, bla, bla, ...

\problem
Bla, bla, bla, bla, bla, ...
\end{tcblisting}


Ci-dessous, il y a ce que l'on obtient si notre sujet ne contient qu'un seul problème et un seul bonus : pas besoin ici d'avoir un numéro pour ces derniers. Pas mal ! Non ?

\resetallcnt{}

\begin{tcblisting}{}
\exercise
Bla, bla, bla, bla, bla, ...

\exercise
Bla, bla, bla, bla, bla, ...

\problem
Bla, bla, bla, bla, bla, ...

\bonus
Bla, bla, bla, bla, bla, ...
\end{tcblisting}



    \subsection{Indiquer le barème}

La commande \verb+\exercise+ dispose de plusieurs options dont l'une permet de donner le nombre de points attribués à une exercice (en indiquant $0$ point, le package indiquera une exercice non noté : voir la section suivante pour un exemple). Voici comment indiquer un barème.

\resetallcnt{}

\begin{tcblisting}{}
\exercise[pts = 5]
Bla, bla, bla, bla, bla, ...

\exercise[pts = 15]
Bla, bla, bla, bla, bla, ...
\end{tcblisting}


    \subsection{Toutes les options en action}

Les fiches techniques données un peu plus bas expliquent le champs d'utilisation de chaque option.

\begin{tcblisting}{}
\exercise[pts   = 0,
          time  = 3 jours,
          id    = facultatif,
          title = Devinette,
          note  = Pour spécialistes uniquement,
          src   = Le livre des experts]

Bla, bla, bla, bla, bla, ...
\end{tcblisting}



    \subsection{Cacher le texte "Exercice"}

La version étoilée \verb+\exercise*+ permet de cacher le contexte qui du point de vue de \verb+lyxam+ sera ici le texte "Exercice". Ceci est surtout pratique pour les sous-parties présentées un peu plus bas. Dans ce genre de situation, il faut obligatoirement donner un titre.
 
\begin{tcblisting}{}
\exercise*[title = Juste mon titre]

Bla, bla, bla, bla, bla, ...
\end{tcblisting}

    \subsection{Fiches techniques}

Toutes les options données ci-dessous sont facultatives.

\bigskip


% IDmacro - All kind of level 2 contexts - START
\IDmacro{activity}{6}{0}

\IDmacro{bonus}{6}{0}

\IDmacro{exercise}{6}{0}

\IDmacro{mcq}{6}{0}

\IDmacro{praticalwork}{6}{0}

\IDmacro{problem}{6}{0}
% IDmacro - All kind of level 2 contexts - END

\exosoptions{}



\section{Indiquer des sous-parties dans vos exercices}

    \subsection{Un exemple suffit\dots{} ou presque}

La commande \verb+\subpart+, qui possède les mêmes options que les commandes de type exercice, permet d'indiquer des sous-parties dans un exercice. La numérotation des sous-parties est remise à zéro à chaque nouvelle utilisation d'une commande de type exercice comme le montre l'exemple suivant. 

\resetallcnt{}

\begin{tcblisting}{}
\exercise
\subpart
Bla, bla, bla, bla, bla, ...
\subpart
Bla, bla, bla, bla, bla, ...

\exercise
\subpart
Bla, bla, bla, bla, bla, ...
\subpart
Bla, bla, bla, bla, bla, ...
\end{tcblisting}


    \subsection{Fiche technique}

Toutes les options données ci-dessous sont facultatives et identiques à celles vues précédemment.

\bigskip


% IDmacro - All kind of level 3 contexts - START
\IDmacro{subpart}{6}{0}
% IDmacro - All kind of level 3 contexts - END

\exosoptions{}



\section{Regrouper vos exercices par thèmes}

    \subsection{Un exemple suffit\dots{} ou presque}

La commande \verb+\topic+, qui possède les mêmes options que les commandes de type exercice, sert juste à regrouper des exercices par thématiques : fût un temps où dans les brevets du collège, il y avait une partie numérique et une autre géométrique.
Chaque nouvelle utilisation de \verb+\topic+ remet juste à zéro la numérotation des sous-parties, et non celles des contextes de type exercice. 

\resetallcnt{}

\resetallcnt{}

\begin{tcblisting}{}
\topic
\exercise
Bla, bla, bla, bla, bla, ...
\exercise
Bla, bla, bla, bla, bla, ...

\topic
\exercise
Bla, bla, bla, bla, bla, ...
\end{tcblisting}


    \subsection{Fiche technique}

Toutes les options données ci-dessous sont facultatives et identiques à celles vues précédemment.

\bigskip


% IDmacro - All kind of level 1 contexts - START
\IDmacro{topic}{6}{0}
% IDmacro - All kind of level 1 contexts - END

\exosoptions{}

\end{document}
