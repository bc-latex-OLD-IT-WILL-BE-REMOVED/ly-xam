\documentclass[12pt,a4paper]{scrartcl}

\usepackage[utf8]{inputenc}
\usepackage[T1]{fontenc}

\usepackage[french]{babel,varioref}

\usepackage{multicol}
\usepackage{subfig}

\usepackage{enumitem}

\usepackage[x11names]{xcolor}
\usepackage{hyperref}
\hypersetup{
    colorlinks,
    citecolor=black,
    filecolor=black,
    linkcolor=black,
    urlcolor=black
}

\usepackage{amsthm}

\usepackage[most]{tcolorbox}
\tcbuselibrary{listingsutf8}

\usepackage{ifplatform}

\usepackage{xstring}

\usepackage{fancyvrb}


% MISC

\tcbset{%
	sharp corners,%
	left=1mm, right=1mm,%
	bottom=1mm, top=1mm,%
	colupper=red!75!blue%
}

\makeatletter
    \newcommand\@example@start@end[2]{%%
        \par\smallskip
        \begingroup%%
            \centering%%
            \setlength{\fboxrule}{0.7pt}%%
            \rule[0.4ex]{#2}{0.7pt}%%
            \framebox{\footnotesize\vphantom{pE}Mise en forme - #1}%%
            \rule[0.4ex]{#2}{0.7pt}%%
            \par\smallskip
        \endgroup%%
    }

    \newcommand\examplestart{
        \color{-red!75!green!50}
        \@example@start@end{Début}{5em}
    }

    \newcommand\exampleend{
        \@example@start@end{Fin}{5.5em}
        \color{black}
    }
\makeatother

\setlength{\parindent}{0cm}

\theoremstyle{definition}
\newtheorem*{remark*}{Remarque}
\newtheorem{remark}{Remarque}

\usepackage[raggedright]{titlesec}

\titleformat{\paragraph}[hang]{\normalfont\normalsize\bfseries}{\theparagraph}{1em}{}
\titlespacing*{\paragraph}{0pt}{3.25ex plus 1ex minus .2ex}{0.5em}


\makeatletter
    \newcommand\resetallcnt{
    	\setcounter{lyxam@counter@topic}{0}
    	\setcounter{lyxam@counter@exercise}{0}
    	\setcounter{lyxam@counter@problem}{0}
    	\setcounter{lyxam@counter@bonus}{0}
    	\setcounter{lyxam@counter@subpart}{0}
    }
\makeatother


% Technical IDs

\newwrite\tempfile
\immediate\openout\tempfile=x-\jobname.macros-x.txt
\AtEndDocument{\immediate\closeout\tempfile}


\newcommand\IDconstant[1]{%
    \immediate\write\tempfile{constant@#1}%
}


\makeatletter
\newcommand\IDmacro{\@ifstar{\@IDmacroStar}{\@IDmacroNoStar}}

\newcommand\@IDmacroNoStar[3]{%
    \texttt{%
    	\textbackslash#1%
    	\IfStrEq{#2}{0}{}{%
    		\,\,[#2 Option%
			\IfStrEq{#2}{1}{}{s}]%
		}%
	    \IfStrEq{#3}{}{}{%
    		\,\,(#3 Argument%
			\IfStrEq{#3}{1}{}{s})%
		}
   	}
    \immediate\write\tempfile{macro@#1@#2@#3}%
}

\newcommand\@IDmacroStar[2]{%
    \@IDmacroNoStar{#1}{0}{#2}%
}


\newcommand\IDenv{\@ifstar{\@IDenvStar}{\@IDenvNoStar}}

\newcommand\@IDenvNoStar[3]{%
    \texttt{%
    	\textbackslash#1%
    	\IfStrEq{#2}{0}{}{%
    		\,\,[#2 Option%
			\IfStrEq{#2}{1}{}{s}]%
		}%
	    \IfStrEq{#3}{}{}{%
    		\,\,(#3 Argument%
			\IfStrEq{#3}{1}{}{s})%
		}
   	}
    \immediate\write\tempfile{env@#1@#2@#3}%
}

\newcommand\@IDenvStar[2]{%
    \@IDenvNoStar{#1}{0}{#2}%
}


\newcommand\@IDoptarg{\@ifstar{\@IDoptargStar}{\@IDoptargNoStar}}

\newcommand\@IDoptargStar[2]{%
	\vspace{0.5em}
	--- \texttt{#1%
		\IfStrEq{#2}{}{:}{\,#2:}%
	}%
}

\newcommand\@IDoptargNoStar[2]{%
	\IfStrEq{#2}{}{%
		\@IDoptargStar{#1}{}%
	}{%
		\@IDoptargStar{#1}{\##2}%
	}%
}


\newcommand\IDkey[1]{%
	\@IDoptarg*{Option}{{\itshape "#1"}}%
}


\newcommand\IDoption[1]{%
	\@IDoptarg{Option}{#1}%
}


\newcommand\IDarg[1]{%
	\@IDoptarg{Argument}{#1}%
}
\makeatother



\begin{document}

\section{Historique}

Tous les changements sont décrits en anglais uniquement dans le dossier \verb+change_log+ : voir le code source de \verb+lyxam+ sur \verb+github+. Nous ne donnons ici qu'un très bref historique de \verb+lyxam+.

\begin{description}[leftmargin=1em]
	\setlength\itemsep{1em}

	\item[À SUIVRE !] Nouvelle version à venir...
	\begin{itemize}
        \item Les options de package \verb+short+ et \verb+noshort+ permettent d'indiquer ou non en abrégé, si possible, les types d'exercices, les points et le temps.

        \item Les options de package \verb+about+, valeur par défaut, et \verb+noabout+ feront afficher ou non les informations complémentaires sur les thèmes, les exercices et les parties.

        \item L'option \verb+render+ de la macro \verb+\exam+ a été renommée \verb+deliver+ (ce qui semble plus correct).
        
        \item Ajout de deux options à l'environnement \verb+\preamble+ : l'une pour centrer ou non le contenu, et l'autre pour choisir la largeur occupée par le préambule.

        \item Il est maintenant possible de personnaliser la numérotation des exercices.


        \item \dots
	\end{itemize}

	\item[2017-11-12] Nouvelle version mineure \verb+0.1.0-beta+ du package.
	\begin{itemize}
        \item Les nouvelles options \verb+hf+ et \verb+nohf+ permettent au chargement du package de montrer ou cacher les en-têtes et les pieds de page.

        \item L'option \verb+preamble+ de la macro \verb+\exam+ disparait pour laisser place au nouvel environnement \verb+\begin{preamble}...\end{preamble}+ utilisable n'importe où.

        \item Pour la macro \verb+\exam+, tous les paramètres sont optionnels.

        \item Pour les macros du type \verb+\exercise+, le paramètre \verb+note+ a été renommé \verb+about+.

		\item En interne, l'utilisation de \verb+simplekv+ a permis une refonte complète du code en simplifiant la méthode à utiliser pour créer de nouvelles mises en page "maison".
	\end{itemize}

	\item[2017-11-03] Première version publique \verb+0.0.0-beta+ du package.
\end{description}

\end{document}
