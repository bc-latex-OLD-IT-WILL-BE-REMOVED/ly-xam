\documentclass[12pt,a4paper]{scrartcl}

\usepackage[utf8]{inputenc}
\usepackage[T1]{fontenc}

\usepackage[french]{babel,varioref}

\usepackage{multicol}
\usepackage{subfig}

\usepackage{enumitem}

\usepackage[x11names]{xcolor}
\usepackage{hyperref}
\hypersetup{
    colorlinks,
    citecolor=black,
    filecolor=black,
    linkcolor=black,
    urlcolor=black
}

\usepackage{amsthm}

\usepackage[most]{tcolorbox}
\tcbuselibrary{listingsutf8}

\usepackage{ifplatform}

\usepackage{xstring}

\usepackage{fancyvrb}


% MISC

\tcbset{%
	sharp corners,%
	left=1mm, right=1mm,%
	bottom=1mm, top=1mm,%
	colupper=red!75!blue%
}

\makeatletter
    \newcommand\@example@start@end[2]{%%
        \par\smallskip
        \begingroup%%
            \centering%%
            \setlength{\fboxrule}{0.7pt}%%
            \rule[0.4ex]{#2}{0.7pt}%%
            \framebox{\footnotesize\vphantom{pE}Mise en forme - #1}%%
            \rule[0.4ex]{#2}{0.7pt}%%
            \par\smallskip
        \endgroup%%
    }

    \newcommand\examplestart{
        \color{-red!75!green!50}
        \@example@start@end{Début}{5em}
    }

    \newcommand\exampleend{
        \@example@start@end{Fin}{5.5em}
        \color{black}
    }
\makeatother

\setlength{\parindent}{0cm}

\theoremstyle{definition}
\newtheorem*{remark*}{Remarque}
\newtheorem{remark}{Remarque}

\usepackage[raggedright]{titlesec}

\titleformat{\paragraph}[hang]{\normalfont\normalsize\bfseries}{\theparagraph}{1em}{}
\titlespacing*{\paragraph}{0pt}{3.25ex plus 1ex minus .2ex}{0.5em}


\makeatletter
    \newcommand\resetallcnt{
    	\setcounter{lyxam@counter@topic}{0}
    	\setcounter{lyxam@counter@exercise}{0}
    	\setcounter{lyxam@counter@problem}{0}
    	\setcounter{lyxam@counter@bonus}{0}
    	\setcounter{lyxam@counter@subpart}{0}
    }
\makeatother


% Technical IDs

\newwrite\tempfile
\immediate\openout\tempfile=x-\jobname.macros-x.txt
\AtEndDocument{\immediate\closeout\tempfile}


\newcommand\IDconstant[1]{%
    \immediate\write\tempfile{constant@#1}%
}


\makeatletter
\newcommand\IDmacro{\@ifstar{\@IDmacroStar}{\@IDmacroNoStar}}

\newcommand\@IDmacroNoStar[3]{%
    \texttt{%
    	\textbackslash#1%
    	\IfStrEq{#2}{0}{}{%
    		\,\,[#2 Option%
			\IfStrEq{#2}{1}{}{s}]%
		}%
	    \IfStrEq{#3}{}{}{%
    		\,\,(#3 Argument%
			\IfStrEq{#3}{1}{}{s})%
		}
   	}
    \immediate\write\tempfile{macro@#1@#2@#3}%
}

\newcommand\@IDmacroStar[2]{%
    \@IDmacroNoStar{#1}{0}{#2}%
}


\newcommand\IDenv{\@ifstar{\@IDenvStar}{\@IDenvNoStar}}

\newcommand\@IDenvNoStar[3]{%
    \texttt{%
    	\textbackslash#1%
    	\IfStrEq{#2}{0}{}{%
    		\,\,[#2 Option%
			\IfStrEq{#2}{1}{}{s}]%
		}%
	    \IfStrEq{#3}{}{}{%
    		\,\,(#3 Argument%
			\IfStrEq{#3}{1}{}{s})%
		}
   	}
    \immediate\write\tempfile{env@#1@#2@#3}%
}

\newcommand\@IDenvStar[2]{%
    \@IDenvNoStar{#1}{0}{#2}%
}


\newcommand\@IDoptarg{\@ifstar{\@IDoptargStar}{\@IDoptargNoStar}}

\newcommand\@IDoptargStar[2]{%
	\vspace{0.5em}
	--- \texttt{#1%
		\IfStrEq{#2}{}{:}{\,#2:}%
	}%
}

\newcommand\@IDoptargNoStar[2]{%
	\IfStrEq{#2}{}{%
		\@IDoptargStar{#1}{}%
	}{%
		\@IDoptargStar{#1}{\##2}%
	}%
}


\newcommand\IDkey[1]{%
	\@IDoptarg*{Option}{{\itshape "#1"}}%
}


\newcommand\IDoption[1]{%
	\@IDoptarg{Option}{#1}%
}


\newcommand\IDarg[1]{%
	\@IDoptarg{Argument}{#1}%
}
\makeatother


\usepackage[fr]{exercises}

\newenvironment{preamble}[1][]{}{}


\begin{document}

\section{Utiliser un préambule}

    \subsection{Un exemple avec toutes les options}

L'environnement \verb+\preamble+ sert à rédiger un ou plusieurs paragraphes en préambule de tout un examen, d'un thème, d'un exercice ou d'une partie.
Il propose deux options : l'une \verb+center+ pour centrer le contenu à l'intérieur du préambule, l'autre \verb+scale+ pour indiquer un coefficient multiplicatif pour la largeur allouée au préambule (ce coefficient est multiplié à \verb+\linewidth+ pour calculer la largeur souhaitée).

\resetallcnt{}

\begin{tcblisting}{listing only}
\begin{preamble}[scale = 0.75, center]
    Cet exercice est indépendant \verb+;-)+. Bla, bla, bla, bla, bla, bla,
    bla, bla, bla, bla, bla, bla, bla, bla, bla, bla, bla, bla, bla\dots
\end{preamble}

\exercise
\begin{preamble}
    Dans cet exercice, tout trace de recherche sera prise en compte à
    condition de ne pas explorer le Pôle Nord quand on s'intéresse
    au Pôle Sud. Quoique\dots
\end{preamble}

Tenter, chercher, trouver\dots
\end{tcblisting}
\examplestarted    % ----- START ----- %
\begin{preamble}[scale = 0.75, center]
    Cet exercice est indépendant \verb+;-)+. Bla, bla, bla, bla, bla, bla,
    bla, bla, bla, bla, bla, bla, bla, bla, bla, bla, bla, bla, bla\dots
\end{preamble}

\exercise
\begin{preamble}
    Dans cet exercice, tout trace de recherche sera prise en compte à
    condition de ne pas explorer le Pôle Nord quand on s'intéresse
    au Pôle Sud. Quoique\dots
\end{preamble}

Tenter, chercher, trouver\dots
\examplefinished   % -----  END  ----- %


    \subsection{Fiches techniques}

\IDenv{preamble}{2}{}

\IDkey{scale} un nombre, valant $1$ par défaut, qui sera multiplié à la largeur d'une ligne pour obtenir la largeur souhaitée pour le préambule.

\IDkey{center} un booléen, valant \verb+false+ par défaut, pour centrer ou non le contenu du préambule (notez que \verb+center+ est un raccourci de \verb+center = true+).


\end{document}
