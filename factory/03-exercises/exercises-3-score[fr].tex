\documentclass[12pt,a4paper]{scrartcl}

\usepackage[utf8]{inputenc}
\usepackage[T1]{fontenc}

\usepackage[french]{babel,varioref}

\usepackage{multicol}
\usepackage{subfig}

\usepackage{enumitem}

\usepackage[x11names]{xcolor}
\usepackage{hyperref}
\hypersetup{
    colorlinks,
    citecolor=black,
    filecolor=black,
    linkcolor=black,
    urlcolor=black
}

\usepackage{amsthm}

\usepackage[most]{tcolorbox}
\tcbuselibrary{listingsutf8}

\usepackage{ifplatform}

\usepackage{xstring}

\usepackage{fancyvrb}


% MISC

\tcbset{%
	sharp corners,%
	left=1mm, right=1mm,%
	bottom=1mm, top=1mm,%
	colupper=red!75!blue%
}

\makeatletter
    \newcommand\@example@start@end[2]{%%
        \par\smallskip
        \begingroup%%
            \centering%%
            \setlength{\fboxrule}{0.7pt}%%
            \rule[0.4ex]{#2}{0.7pt}%%
            \framebox{\footnotesize\vphantom{pE}Mise en forme - #1}%%
            \rule[0.4ex]{#2}{0.7pt}%%
            \par\smallskip
        \endgroup%%
    }

    \newcommand\examplestart{
        \color{-red!75!green!50}
        \@example@start@end{Début}{5em}
    }

    \newcommand\exampleend{
        \@example@start@end{Fin}{5.5em}
        \color{black}
    }
\makeatother

\setlength{\parindent}{0cm}

\theoremstyle{definition}
\newtheorem*{remark*}{Remarque}
\newtheorem{remark}{Remarque}

\usepackage[raggedright]{titlesec}

\titleformat{\paragraph}[hang]{\normalfont\normalsize\bfseries}{\theparagraph}{1em}{}
\titlespacing*{\paragraph}{0pt}{3.25ex plus 1ex minus .2ex}{0.5em}


\makeatletter
    \newcommand\resetallcnt{
    	\setcounter{lyxam@counter@topic}{0}
    	\setcounter{lyxam@counter@exercise}{0}
    	\setcounter{lyxam@counter@problem}{0}
    	\setcounter{lyxam@counter@bonus}{0}
    	\setcounter{lyxam@counter@subpart}{0}
    }
\makeatother


% Technical IDs

\newwrite\tempfile
\immediate\openout\tempfile=x-\jobname.macros-x.txt
\AtEndDocument{\immediate\closeout\tempfile}


\newcommand\IDconstant[1]{%
    \immediate\write\tempfile{constant@#1}%
}


\makeatletter
\newcommand\IDmacro{\@ifstar{\@IDmacroStar}{\@IDmacroNoStar}}

\newcommand\@IDmacroNoStar[3]{%
    \texttt{%
    	\textbackslash#1%
    	\IfStrEq{#2}{0}{}{%
    		\,\,[#2 Option%
			\IfStrEq{#2}{1}{}{s}]%
		}%
	    \IfStrEq{#3}{}{}{%
    		\,\,(#3 Argument%
			\IfStrEq{#3}{1}{}{s})%
		}
   	}
    \immediate\write\tempfile{macro@#1@#2@#3}%
}

\newcommand\@IDmacroStar[2]{%
    \@IDmacroNoStar{#1}{0}{#2}%
}


\newcommand\IDenv{\@ifstar{\@IDenvStar}{\@IDenvNoStar}}

\newcommand\@IDenvNoStar[3]{%
    \texttt{%
    	\textbackslash#1%
    	\IfStrEq{#2}{0}{}{%
    		\,\,[#2 Option%
			\IfStrEq{#2}{1}{}{s}]%
		}%
	    \IfStrEq{#3}{}{}{%
    		\,\,(#3 Argument%
			\IfStrEq{#3}{1}{}{s})%
		}
   	}
    \immediate\write\tempfile{env@#1@#2@#3}%
}

\newcommand\@IDenvStar[2]{%
    \@IDenvNoStar{#1}{0}{#2}%
}


\newcommand\@IDoptarg{\@ifstar{\@IDoptargStar}{\@IDoptargNoStar}}

\newcommand\@IDoptargStar[2]{%
	\vspace{0.5em}
	--- \texttt{#1%
		\IfStrEq{#2}{}{:}{\,#2:}%
	}%
}

\newcommand\@IDoptargNoStar[2]{%
	\IfStrEq{#2}{}{%
		\@IDoptargStar{#1}{}%
	}{%
		\@IDoptargStar{#1}{\##2}%
	}%
}


\newcommand\IDkey[1]{%
	\@IDoptarg*{Option}{{\itshape "#1"}}%
}


\newcommand\IDoption[1]{%
	\@IDoptarg{Option}{#1}%
}


\newcommand\IDarg[1]{%
	\@IDoptarg{Argument}{#1}%
}
\makeatother


\usepackage[fr, score, book]{exercises}


\begin{document}

\section{Indiquer un barème détaillé} \label{exercises:score}

    \subsection{Ce qui est visible}

Pour indiquer le barème d'un devoir, il faut utiliser la macro \verb+\scpts+.
Le cas le plus basique consiste à juste indiquer le nombre maximum de points obtenus pour un morceau de réponse comme dans l'exemple suivant mais nous allons voir que \verb+\scpts+ ne fait pas que de l'affichage. Notez au passage que l'on utilise le point comme séparateur décimal !

\begin{tcblisting}{listing only}
\scpts{1} Question 1 : bla, bla, bla, bla, bla, bla, bla, bla, bla, bla,
bla, bla, bla, bla, bla, bla, bla, bla, bla, bla, bla, bla, bla, bla, bla,
bla, bla, bla, bla, bla, bla, bla, bla, bla\dots

\scpts{0.5} Question 2 : bla, bla, bla, bla, bla, bla, bla, bla, bla, bla, 
bla, bla, bla, bla, bla, bla, bla, bla, bla, bla, bla, bla, bla, bla, bla,
bla, bla, bla, bla, bla, bla, bla, bla, bla\dots

\scpts{1.5} Question 3 : bla, bla, bla, bla, bla, bla, bla, bla, bla, bla, 
bla, bla, bla, bla, bla, bla, bla, bla, bla, bla, bla, bla, bla, bla, bla,
bla, bla, bla, bla, bla, bla, bla, bla, bla\dots
\end{tcblisting}

\examplestart{}
\scpts{1} Question 1 : bla, bla, bla, bla, bla, bla, bla, bla, bla, bla,
bla, bla, bla, bla, bla, bla, bla, bla, bla, bla, bla, bla, bla, bla, bla,
bla, bla, bla, bla, bla, bla, bla, bla, bla\dots

\scpts{0.5} Question 2 : bla, bla, bla, bla, bla, bla, bla, bla, bla, bla, 
bla, bla, bla, bla, bla, bla, bla, bla, bla, bla, bla, bla, bla, bla, bla,
bla, bla, bla, bla, bla, bla, bla, bla, bla\dots

\scpts{1.5} Question 3 : bla, bla, bla, bla, bla, bla, bla, bla, bla, bla, 
bla, bla, bla, bla, bla, bla, bla, bla, bla, bla, bla, bla, bla, bla, bla,
bla, bla, bla, bla, bla, bla, bla, bla, bla\dots
\exampleend{}



    \subsection{Tout ce que l'on peut détailler}

La macro \verb+\scpts+ possède un argument optionnel et un argument obligatoire qui s'utilisent comme suit.
\begin{enumerate}
	\item L'argument obligatoire permet non seulement d'indiquer le nombre maximal de points mais aussi si besoin différents sous-scores.
	Par exemple, \verb+\scpts{1.5,1,0.5}+ indique que l'on peut avoir soit $1,\!5$ points, soit $1$ point, soit $0,\!5$ points.

	\item L'argument optionnel permet d'indiquer brièvement ce qui est évalué.
	Par exemple, \verb&\scpts[Développement de $(a + b)^2$]{1}& précise que le développement de $(a + b)^2$ rapporte $1$ point sans sous-score possible.
\end{enumerate}


\medskip

À quoi bon donner autant d'informations ? Le plus simple est de considérer un exemple.
Nous supposons que le fichier de chemin \verb+/Users/login/test_scores.tex+ ne contient qu'une seule utilisation de la macro \verb+\scpts+ qui est la suivante.

\begin{tcblisting}{listing only}
\exercise

\scpts[$(a + b)^2$]{1,0.75,0.5} Bla, bla, bla, bla, bla, bla, bla, bla, bla,
bla, bla, bla, bla, bla, bla, bla, bla, bla\dots
\end{tcblisting}


Une fois le fichier \LaTeX{} compilé, un fichier \verb+/Users/login/test_scores.scores+ sera fabriqué par le package \verb+lyxam+ avec le contenu suivant.

\begin{tcblisting}{listing only}
// context=kind::nb::id::title::points
// ---> Note: an empty value is indicated by {}
// eval=short text describing what is evaluated
// scores=max::optional subscore 1::optional subscore 2::...
context=exercise::1::{}::{}::{}
eval=$(a + b)^2$
scores=1::0.75::0.5
\end{tcblisting}

Ce fichier est destiné à être utilisé par un programme tierce (on peut par exemple créer une feuille de calcul facilitant la correction de copies). La syntaxe utilisée est la suivante.
\begin{enumerate}
	\item Les lignes commençant par \verb+//+ sont des commentaires.

	\item Une ligne du type \verb+context=kind::nb::id::title::points+ donne des informations sur un nouveau contexte rencontré, un contexte étant un thème, un exercice, une sous-partie\dots 
	\begin{itemize}
		\item \verb+kind+ donne le type de contexte. Voici les valeurs possibles.

% All kinds of contexts - START
        \begin{center}
            \begin{tabular}{|ll@{\hskip 0.5ex}|l@{\hskip 0.5ex}l|}
                \hline activity &&& praticalwork \\
                \hline bonus &&& problem \\
                \hline exercise &&& subpart \\
                \hline mcq &&& topic \\
                \hline
            \end{tabular}
        \end{center}
% All kinds of contexts - END

		\item \verb+nb+, \verb+id+ et \verb+title+ donnent respectivement le numéro du contexte, son identifiant personnalisé et son titre, chaque valeur pouvant prendre la valeur textuelle \verb+{}+ qui indique une valeur vide.

		\item \verb+points+ correspond au nombre de points attribués à un contexte.
	\end{itemize}

	\item \verb+eval=...+ indique ce qui est évalué. Si rien n'est indiqué dans le fichier \LaTeX{}, le fichier \verb+.scores+ contiendra \verb+eval=+ et non \verb+eval={}+ (désolé pour ce manque de cohérence\dots).
	
	\item \verb+scores=1::0.75::0.5+ donne le score total suivis de sous-scores, avec autant de sous-scores que souhaités.
\end{enumerate}


    \subsection{Fiche technique}

\IDmacro{scpts}{1}{1}

\IDoption{} description courte de ce qui est évalué.

\IDarg{} le nombre maximal de points avec le point pour séparateur décimal, puis éventuellement différents sous-scores séparés par des virgules.

\end{document}
