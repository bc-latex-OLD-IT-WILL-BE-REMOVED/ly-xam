%  ---------------------
%  SPECIFIC FRENCH EXAMS  (no translation)
%  ---------------------

% We add new kind of exams and optional arguments
% so as to indicate exercises for all candidates,
% specialists or no-specialists.

\renewcommand{\@moreKindsOfExams}{bac,bacBlanc,brevet,brevetBlanc}
\renewcommand{\@optionalArgForExeOrPb}{spe,noSpe,all}

%  +++
%  BAC
%  +++

\newcommand{\bacText}{Baccalauréat}
\newcommand{\bacCorText}{Corrigé du Baccalauréat}

\newcommand{\bacBlancText}{Baccalauréat Blanc}
\newcommand{\bacBlancCorText}{Corrigé du Baccalauréat Blanc}

\newcommand{\bacPreambule}{%
	Le candidat est invité à faire figurer sur sa copie toute trace de
	recherche, même incomplète ou non fructueuse, qu'il aura développée.
	\\
	Il est rappelé aux candidats que la qualité de la rédaction,
	la clarté et la précision des raisonnements entreront pour
	une part importante dans l'appréciation des copies. %
}

\newcommand{\allExerciseText}{Commun à tous les candidats}
\newcommand{\speExerciseText}{Candidats ayant suivi l’enseignement de spécialité}
\newcommand{\noSpeExerciseText}{Candidats n'ayant pas suivi l’enseignement de spécialité}


%  +++++++++++++++++++
%  Brevet des collèges
%  +++++++++++++++++++

\newcommand{\brevetText}{Brevet des collèges}
\newcommand{\brevetCorText}{Corrigé du Brevet des collèges}

\newcommand{\brevetBlancText}{Brevet blanc}
\newcommand{\brevetBlancCorText}{Corrigé du Brevet blanc}

\newcommand{\brevetPreambule}{%
	Toutes les réponses doivent être justifiées, sauf si une indication
	contraire est donnée. %
}


%  --------------
%  STANDARD EXAMS
%  --------------

\newcommand{\testText}{Devoir Surveillé}
\newcommand{\testCorText}{Corrigé du Devoir Surveillé}

\newcommand{\homeText}{Devoir à la Maison}
\newcommand{\homeCorText}{Corrigé du Devoir à la Maison}

\newcommand{\shortText}{Interrogation rapide}
\newcommand{\shortCorText}{Corrigé de l'Interrogation rapide}

\newcommand{\noCalcPreambule}{%
	Dans ce devoir, l'usage de la calculatrice est strictement interdit. %
}


%  -----------------------
%  SPECIAL TEXTS FOR EXAMS
%  -----------------------

\newcommand{\timeText}{Durée}
\newcommand{\handBackText}{Prénom NOM}
\newcommand{\nbText}{N$^\circ$}


%  ---------------------------
%  SPECIAL TEXTS FOR EXERCISES
%  ---------------------------

\newcommand{\numericActivityText}{Activité Numérique}
\newcommand{\geometricActivityText}{Activité Géométrique}
\newcommand{\algebricActivityText}{Activité Algébrique}

\newcommand{\exerciseText}{Exercice}
\newcommand{\problemText}{Problème}
\newcommand{\bonusText}{Bonus}
\newcommand{\subPartText}{Partie}


%  --------------------
%  TRUE-FALSE QUESTIONS
%  --------------------

\newcommand{\propText}{Proposition}
\newcommand{\trueText}{Vrai}
\newcommand{\falseText}{Faux}
\newcommand{\trueFemText}{Vraie}
\newcommand{\falseFemText}{Fausse}


%  -----------------------
%  GIVING SOME INDICATIONS
%  -----------------------

\newcommand{\roc}{%  (no translation)
	Restitution organisée de connaissances%
}
\newcommand{\openedQuestionText}{%
	Dans cette question toute trace de recherche, même incomplète,
	sera prise en compte dans l'évaluation.%
}







%Pour chacune des propositions suivantes, indiquer si elle est vraie oufause et donner ne démonstration de la réponse choisie. Dans le cas d’une proposition fausse, la démonstration consistera à proposer un contre-exemple. Une réponse non démontrée ne rapporte aucun point.







\newcommand{\justifyText}{%
	Vous justifierez chacune de vos réponses.%
}
\newcommand{\justifyHandBackText}{%
	Directement sur l'énoncé, vous justifierez chacune de vos réponses.%
}


\newcommand{\choiceSingleText}{%
	Sur votre feuille, donnez la bonne réponse dans chacun des cas suivants.%
}
\newcommand{\choiceSingleHandBackText}{%
	Directement sur l'énoncé, cochez la bonne réponse dans chacun des cas suivants.%
}
\newcommand{\choiceMutlipleText}{%
	Sur votre feuille, donnez la ou les bonnes réponses dans chacun des cas suivants.%
}
\newcommand{\choiceMutlipleHandBackText}{%
	Directement sur l'énoncé, cochez la ou les bonnes réponses dans chacun des cas suivants.%
}


\newcommand{\truegggFalseText}{%
	Sur votre feuille, répondez par VRAI ou FAUX dans chacun des cas suivants.%
}
\newcommand{\trueFalseHandBackText}{%
	Directement sur l'énoncé, répondez par VRAI ou FAUX dans chacun des cas suivants.%
}
