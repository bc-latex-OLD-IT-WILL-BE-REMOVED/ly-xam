\documentclass[12pt,a4paper]{scrartcl}

\usepackage[utf8]{inputenc}
\usepackage[T1]{fontenc}

\usepackage[french]{babel,varioref}

\usepackage{multicol}
\usepackage{subfig}

\usepackage{enumitem}

\usepackage[x11names]{xcolor}
\usepackage{hyperref}
\hypersetup{
    colorlinks,
    citecolor=black,
    filecolor=black,
    linkcolor=black,
    urlcolor=black
}

\usepackage{amsthm}

\usepackage[most]{tcolorbox}
\tcbuselibrary{listingsutf8}

\usepackage{ifplatform}

\usepackage{xstring}

\usepackage{fancyvrb}


% MISC

\tcbset{%
	sharp corners,%
	left=1mm, right=1mm,%
	bottom=1mm, top=1mm,%
	colupper=red!75!blue%
}

\makeatletter
    \newcommand\@example@start@end[2]{%%
        \par\smallskip
        \begingroup%%
            \centering%%
            \setlength{\fboxrule}{0.7pt}%%
            \rule[0.4ex]{#2}{0.7pt}%%
            \framebox{\footnotesize\vphantom{pE}Mise en forme - #1}%%
            \rule[0.4ex]{#2}{0.7pt}%%
            \par\smallskip
        \endgroup%%
    }

    \newcommand\examplestart{
        \color{-red!75!green!50}
        \@example@start@end{Début}{5em}
    }

    \newcommand\exampleend{
        \@example@start@end{Fin}{5.5em}
        \color{black}
    }
\makeatother

\setlength{\parindent}{0cm}

\theoremstyle{definition}
\newtheorem*{remark*}{Remarque}
\newtheorem{remark}{Remarque}

\usepackage[raggedright]{titlesec}

\titleformat{\paragraph}[hang]{\normalfont\normalsize\bfseries}{\theparagraph}{1em}{}
\titlespacing*{\paragraph}{0pt}{3.25ex plus 1ex minus .2ex}{0.5em}


\makeatletter
    \newcommand\resetallcnt{
    	\setcounter{lyxam@counter@topic}{0}
    	\setcounter{lyxam@counter@exercise}{0}
    	\setcounter{lyxam@counter@problem}{0}
    	\setcounter{lyxam@counter@bonus}{0}
    	\setcounter{lyxam@counter@subpart}{0}
    }
\makeatother


% Technical IDs

\newwrite\tempfile
\immediate\openout\tempfile=x-\jobname.macros-x.txt
\AtEndDocument{\immediate\closeout\tempfile}


\newcommand\IDconstant[1]{%
    \immediate\write\tempfile{constant@#1}%
}


\makeatletter
\newcommand\IDmacro{\@ifstar{\@IDmacroStar}{\@IDmacroNoStar}}

\newcommand\@IDmacroNoStar[3]{%
    \texttt{%
    	\textbackslash#1%
    	\IfStrEq{#2}{0}{}{%
    		\,\,[#2 Option%
			\IfStrEq{#2}{1}{}{s}]%
		}%
	    \IfStrEq{#3}{}{}{%
    		\,\,(#3 Argument%
			\IfStrEq{#3}{1}{}{s})%
		}
   	}
    \immediate\write\tempfile{macro@#1@#2@#3}%
}

\newcommand\@IDmacroStar[2]{%
    \@IDmacroNoStar{#1}{0}{#2}%
}


\newcommand\IDenv{\@ifstar{\@IDenvStar}{\@IDenvNoStar}}

\newcommand\@IDenvNoStar[3]{%
    \texttt{%
    	\textbackslash#1%
    	\IfStrEq{#2}{0}{}{%
    		\,\,[#2 Option%
			\IfStrEq{#2}{1}{}{s}]%
		}%
	    \IfStrEq{#3}{}{}{%
    		\,\,(#3 Argument%
			\IfStrEq{#3}{1}{}{s})%
		}
   	}
    \immediate\write\tempfile{env@#1@#2@#3}%
}

\newcommand\@IDenvStar[2]{%
    \@IDenvNoStar{#1}{0}{#2}%
}


\newcommand\@IDoptarg{\@ifstar{\@IDoptargStar}{\@IDoptargNoStar}}

\newcommand\@IDoptargStar[2]{%
	\vspace{0.5em}
	--- \texttt{#1%
		\IfStrEq{#2}{}{:}{\,#2:}%
	}%
}

\newcommand\@IDoptargNoStar[2]{%
	\IfStrEq{#2}{}{%
		\@IDoptargStar{#1}{}%
	}{%
		\@IDoptargStar{#1}{\##2}%
	}%
}


\newcommand\IDkey[1]{%
	\@IDoptarg*{Option}{{\itshape "#1"}}%
}


\newcommand\IDoption[1]{%
	\@IDoptarg{Option}{#1}%
}


\newcommand\IDarg[1]{%
	\@IDoptarg{Argument}{#1}%
}
\makeatother



\begin{document}

\section{Les options du package}

Avant d'entrer dans le vif du sujet, nous donnons ici toutes les options utilisables lors de l'appel du package via \verb+\usepackage[...]{lyxam}+.

\begin{itemize}[label=\textbullet]
%	\item \verb+apmep+ indique d'utiliser le style de mise en forme nommé "apmep" en référence aux sujets de BAC proposés sur le site de \href{https://www.apmep.fr}{l'APMEP}, l'Association des Professeurs de Mathématiques de l'Enseignement public. C'est le seul type de mise en forme disponible pour le moment !

	\item \verb+en+, valeur par défaut
	\footnote{
		Sorry for the french frogs that we are...
	}, et \verb+fr+ permettent d'indiquer d'utiliser l'anglais ou le français pour tous les textes ajoutés par \verb+lyxam+.

	\item \verb+hf+, valeur par défaut, et \verb+nohf+ demandent d'afficher ou non les en-têtes et les pieds de page.
	Les lettres \verb+hf+ sont pour "\textbf{h}-eaders" et "\textbf{f}-ooters", soit "en-tête" et "pied de page" en français.

	\item \verb+pts+, valeur par défaut, et \verb+nopts+ servent à voir ou non les points donnés pour chaque exercice.

	\item \verb+noshort+, valeur par défaut, et \verb+short+ demandent d'utiliser le nom long ou abrégé des types d'exercices.

	\item \verb+src+, valeur par défaut, et \verb+nosrc+ feront afficher ou non les sources utilisées.
\end{itemize}

%\begin{remark}
%	Dans la suite de cette documentation, nous utilisons les options \verb+apmep+ et \verb+fr+, c'est à dire via \verb+\usepackage[apmep, fr]{lyxam}+.
%\end{remark}
\begin{remark}
	Dans la suite de cette documentation, nous invoquons juste l'option \verb+fr+.Autrement dit, nous passons par \verb+\usepackage[fr]{lyxam}+.
\end{remark}

\end{document}
