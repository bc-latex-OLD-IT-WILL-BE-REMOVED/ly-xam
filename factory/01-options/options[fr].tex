\documentclass[12pt,a4paper]{article}

% == FOR DOC AND TESTS - START == %

\usepackage[utf8]{inputenc}
\usepackage{ucs}
\usepackage[top=2cm, bottom=2cm, left=1.5cm, right=1.5cm]{geometry}

\usepackage[francais]{babel}

\usepackage{color}
\usepackage{hyperref}
\hypersetup{
    colorlinks,
    citecolor=black,
    filecolor=black,
    linkcolor=black,
    urlcolor=black
}

\usepackage{multicol}
\usepackage{enumitem}

\usepackage{amsthm}

\usepackage{tcolorbox}
\tcbuselibrary{listingsutf8}

\usepackage{pgffor}
\usepackage{xstring}


% MISC

\tcbset{%
	sharp corners,%
	left=1mm, right=1mm,%
	bottom=1mm, top=1mm,%
	colupper=red!75!blue% 
}

\setlength{\parindent}{0cm}

\theoremstyle{definition}
\newtheorem*{remark}{Remarque}

\usepackage[raggedright]{titlesec}

\titleformat{\paragraph}[hang]{\normalfont\normalsize\bfseries}{\theparagraph}{1em}{}
\titlespacing*{\paragraph}{0pt}{3.25ex plus 1ex minus .2ex}{0.5em}

\makeatother
	\newcommand\resetallcnt{
		\setcounter{lyxam@counter@topic}{0}
		\setcounter{lyxam@counter@exercise}{0}
		\setcounter{lyxam@counter@problem}{0}
		\setcounter{lyxam@counter@bonus}{0}
		\setcounter{lyxam@counter@subpart}{0}
	}
\makeatletter

% Technical IDs

\newwrite\tempfile

\immediate\openout\tempfile=x-\jobname.macros-x.txt

\AtEndDocument{\immediate\closeout\tempfile}

\newcommand\IDconstant[1]{%
    \immediate\write\tempfile{constant@#1}%
}

\makeatletter
	\newcommand\IDmacro{\@ifstar{\@IDmacroStar}{\@IDmacroNoStar}}
	
    \newcommand\@IDmacroNoStar[3]{%
        \texttt{%
        	\textbackslash#1%
        	\IfStrEq{#2}{0}{}{%
        		\,\,[#2 Option%
				\IfStrEq{#2}{1}{}{s}]%
			}%
    	    \IfStrEq{#3}{}{}{%
	    		\,\,(#3 Argument%
				\IfStrEq{#3}{1}{}{s})%
			}
	   	}
        \immediate\write\tempfile{macro@#1@#2@#3}%
    }

    \newcommand\@IDmacroStar[2]{%
        \@IDmacroNoStar{#1}{0}{#2}%
    }

	\newcommand\@IDoptarg{\@ifstar{\@IDoptargStar}{\@IDoptargNoStar}}
	
	\newcommand\@IDoptargStar[2]{%
    	\vspace{0.5em}
		--- \texttt{#1%
			\IfStrEq{#2}{}{:}{\,#2:}%
		}%
	}

	\newcommand\@IDoptargNoStar[2]{%
    	\IfStrEq{#2}{}{%
			\@IDoptargStar{#1}{}%
		}{%
			\@IDoptargStar{#1}{\##2}%
		}%
	}

	\newcommand\IDkey[1]{%
    	\@IDoptarg*{Option}{{\itshape "#1"}}%
	}

	\newcommand\IDoption[1]{%
    	\@IDoptarg{Option}{#1}%
	}

	\newcommand\IDarg[1]{%
    	\@IDoptarg{Argument}{#1}%
	}
\makeatother

% == FOR DOC AND TESTS - END == %


\begin{document}

\section{Les options du package}

Avant d'entrer dans le vif du sujet, nous donnons ici toutes les options utilisables lors de l'appel du package via \verb+\usepackage[...]{lyxam}+.

\begin{itemize}[label=\textbullet]
	\item \verb+apmep+ indique d'utiliser le style de mise en forme nommé "apmep" en référence aux sujets de BAC proposés sur le site de \href{https://www.apmep.fr}{l'APMEP}, l'Association des Professeurs de Mathématiques de l'Enseignement public. C'est le seul type de mise en forme disponible pour le moment !

	\item \verb+fr+ permet d'indiquer d'utiliser le français pour tous les textes ajoutés par \verb+lyxam+.

	\verb+en+ permet d'indiquer d'utiliser l'anglais. \textbf{C'est la valeur par défaut.} Sorry for the french frogs that we are...

	\item \verb+src+ ou \verb+nosrc+ demande d'afficher ou non les sources utilisées.

	\item \verb+pts+ ou \verb+nopts+ sert à voir ou non les points des exercices.
\end{itemize}

\begin{remark}
	Dans la suite de cette documentation, nous utilisons \verb+\usepackage[apmep, fr]{lyxam}+.
\end{remark}


\end{document}
