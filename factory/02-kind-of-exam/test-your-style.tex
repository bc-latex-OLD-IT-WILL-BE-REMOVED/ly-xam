\documentclass[12pt,a4paper]{scrartcl}
\usepackage[showframe]{geometry}

\usepackage[utf8]{inputenc}
\usepackage[T1]{fontenc}

\usepackage[%
%            apmep,%
%            linebox,%
			mini,
            fr,%
            %nohf%
]{kind-of-exam}

\usepackage{lipsum}
\usepackage[french]{babel}


\begin{document}

% ============ %
% == TEST 1 == %
% ============ %

\exam[deliver  = short,%
      kind     = D.S.,%
      nb       = 1,%
      subnb    = Subject A,%
      subject  = Mathematics,%
      theme    = Probability \& Functions \& Functions \& Functions \& Functions \& Functions \& Functions \& Functions \& Functions \& Functions \& Functions \& Functions \& Functions \& Functions \& Functions \& Functions,%
      sector   = Scientific Cursus,%
      class    = 1S4,%
      location = Lycée France\,(Paris),%
      date     = 2017-09-23,%
      time     = 4h%
]


\begin{preamble}[%%
    scale = 0.8,
    center %%
]
    +++ \textbf{PREAMBULE: STARTS HERE} +++

    \lipsum[1]

    [...]

    \lipsum[2]

    +++ \textbf{PREAMBULE FINISHES HERE} +++
\end{preamble}


\textbf{EXERCISES START HERE !} \lipsum[1]

\lipsum



% ============ %
% == TEST 2 == %
% ============ %

\newpage

\exam

\textbf{EXERCISES START HERE !} \lipsum[1]

\lipsum



% ============ %
% == TEST 3 == %
% ============ %

\newpage


\exam[deliver  = long,
      kind     = Test,%
      nb       = 2,%
      theme    = Probability \& Functions,%
      class    = 1S4,%
      location = Lycée France (Paris),%
      date     = 2017-11-02,%
      time     = 4h%
]

\lipsum

\end{document}
