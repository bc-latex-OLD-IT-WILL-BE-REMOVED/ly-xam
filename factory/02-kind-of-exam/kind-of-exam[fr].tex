\documentclass[12pt,a4paper]{scrartcl}

\usepackage[apmep, fr]{kind-of-exam}

% == FOR DOC AND TESTS - START == %

\usepackage[utf8]{inputenc}
\usepackage[T1]{fontenc}
\usepackage{ucs}

\usepackage[french]{babel,varioref}

\usepackage{multicol}
\usepackage{subfig}
\usepackage{enumitem}

\usepackage{color}
\usepackage{hyperref}
\hypersetup{
    colorlinks,
    citecolor=black,
    filecolor=black,
    linkcolor=black,
    urlcolor=black
}

\usepackage{amsthm}

\usepackage{tcolorbox}
\tcbuselibrary{listingsutf8}

\usepackage{ifplatform}

\usepackage{xstring}

\usepackage{fancyvrb}


% MISC

\tcbset{%
	sharp corners,%
	left=1mm, right=1mm,%
	bottom=1mm, top=1mm,%
	colupper=red!75!blue%
}

\setlength{\parindent}{0cm}

\theoremstyle{definition}
\newtheorem*{remark}{Remarque}

\usepackage[raggedright]{titlesec}

\titleformat{\paragraph}[hang]{\normalfont\normalsize\bfseries}{\theparagraph}{1em}{}
\titlespacing*{\paragraph}{0pt}{3.25ex plus 1ex minus .2ex}{0.5em}

\makeatother
	\newcommand\resetallcnt{
		\setcounter{lyxam@counter@topic}{0}
		\setcounter{lyxam@counter@exercise}{0}
		\setcounter{lyxam@counter@problem}{0}
		\setcounter{lyxam@counter@bonus}{0}
		\setcounter{lyxam@counter@subpart}{0}
	}
\makeatletter

% Technical IDs

\newwrite\tempfile

\immediate\openout\tempfile=x-\jobname.macros-x.txt

\AtEndDocument{\immediate\closeout\tempfile}

\newcommand\IDconstant[1]{%
    \immediate\write\tempfile{constant@#1}%
}

\makeatletter
	\newcommand\IDmacro{\@ifstar{\@IDmacroStar}{\@IDmacroNoStar}}

    \newcommand\@IDmacroNoStar[3]{%
        \texttt{%
        	\textbackslash#1%
        	\IfStrEq{#2}{0}{}{%
        		\,\,[#2 Option%
				\IfStrEq{#2}{1}{}{s}]%
			}%
    	    \IfStrEq{#3}{}{}{%
	    		\,\,(#3 Argument%
				\IfStrEq{#3}{1}{}{s})%
			}
	   	}
        \immediate\write\tempfile{macro@#1@#2@#3}%
    }

    \newcommand\@IDmacroStar[2]{%
        \@IDmacroNoStar{#1}{0}{#2}%
    }

	\newcommand\@IDoptarg{\@ifstar{\@IDoptargStar}{\@IDoptargNoStar}}

	\newcommand\@IDoptargStar[2]{%
    	\vspace{0.5em}
		--- \texttt{#1%
			\IfStrEq{#2}{}{:}{\,#2:}%
		}%
	}

	\newcommand\@IDoptargNoStar[2]{%
    	\IfStrEq{#2}{}{%
			\@IDoptargStar{#1}{}%
		}{%
			\@IDoptargStar{#1}{\##2}%
		}%
	}

	\newcommand\IDkey[1]{%
    	\@IDoptarg*{Option}{{\itshape "#1"}}%
	}

	\newcommand\IDoption[1]{%
    	\@IDoptarg{Option}{#1}%
	}

	\newcommand\IDarg[1]{%
    	\@IDoptarg{Argument}{#1}%
	}
\makeatother

% == FOR DOC AND TESTS - END == %


\begin{document}

\section{Quel devoir donnez-vous ?}

	\subsection{La commande \texttt{\textbackslash exam}}

La commande \verb+\exam+ permet de donner des informations générales à propos du devoir. 
Le code ci-dessous utilise toutes les options disponibles, et la figure \ref{style:apmep} \vpageref{style:apmep} donne un aperçu du rendu obtenu dans ce cas. 

\begin{tcblisting}{listing only}
\exam[render   = true,%
      kind     = D.S.,%
      nb       = 1,%
      subnb    = Sujet A,%
      subject  = Mathématiques,%
      theme    = Probabilités,%
      sector   = Série Scientifique,%
      class    = 1S4,%
      location = Lycée MONGE (Chambéry),%
      date     = 20/10/2017,%
      time     = 2h]
\end{tcblisting}


\begin{figure}[!tbp]
  \setlength{\fboxrule}{1.5pt}
  \centering
  \fbox{\includegraphics[width=0.47\linewidth]{example-doc[fr]-0.jpg}}
  \hfill
  \fbox{\includegraphics[width=0.47\linewidth]{example-doc[fr]-1.jpg}}
  \caption{Style \texttt{apmep}.}
  \label{style:apmep}
\end{figure}

Expliquons maintenant en détail le rôle de chacun des paramètres qui sont tous optionnels. Lorsqu'aucune valeur par défaut n'est indiquée, c'est que cette valeur est un texte vide.

\begin{enumerate}
	\item \verb+render+, valant \verb+false+ par défaut, est à utiliser pour un sujet à rendre avec la copie : utiliser \verb+false+, la valeur par défaut, ou  \verb+true+ pour \emph{"non"} ou \emph{"oui"}.
	Ceci a pour conséquence que le sujet débutera, ou non, par une zone où l'élève devra indiquer son nom et son prénom.

	\item \verb+kind+, valant \verb+Test+ par défaut, est le type de devoir : un \emph{"D.S."}, un \emph{"D.M."}, une \emph{"Interrogation Surprise"}, une \emph{"Fiche d'entraînement"}, une \emph{"Activité"}...
	Vous noterez que le terme \emph{"devoir"} ne se limite pas juste aux devoirs notés.

	\item \verb+nb+ est le numéro du devoir.

	\item \verb+subnb+ permet d'indiquer une sorte de numérotation secondaire. C'est utile par exemple pour indiquer \emph{"Sujet A"}, \emph{"Sujet B"} ...

	\item \verb+subject+ permet si besoin de donner la thématique générale du devoir comme par exemple \emph{"Mathématiques"}, \emph{"Informatique Générale"}...

	\item \verb+theme+ sert à compléter la thématique générale en indiquant un ou des points particuliers comme par exemple \emph{"Probabilités"}, \emph{"Réseaux"} ...

	\item \verb+sector+ sert à indiquer une section, au sens administratif, à laquelle s'adresse le devoir. Par exemple, pour un sujet de Baccalauréat S en France, on serait amené à utiliser \verb+sector = Série Scientifique+.

	\item \verb+class+ indique la classe et/ou le groupe auquel est destiné le devoir.

	\item \verb+location+ vous permet d'indiquer un lieu géographique, typiquement un établissement scolaire ou universitaire.

	\item \verb+date+ est pour la date du devoir.

	\item \verb+time+ est pour la durée du devoir.
\end{enumerate}


%	\subsection{Différentes mises en forme}
%
%Le package est fourni avec des exemples de fichiers \LaTeX{} afin d'observer ce que permet \verb+lyxam+. Explorez-les !


	\subsection{Fiche technique}

\IDmacro{exam}{11}{}

\IDkey{kind} le type de devoir, la valeur par défaut étant \verb+Test+.

\IDkey{render} pour un sujet à rendre, ou non, avec une zone pour le nom et le prénom de l'étudiant. Deux valeurs possibles : \verb+false+, valeur par défaut, ou \verb+true+.

\IDkey{nb} le numéro du devoir.

\IDkey{subnb} une numérotation secondaire du devoir.

\IDkey{subject} la matière, le sujet du devoir.

\IDkey{theme} un sous-thème ou une sous-partie de la matière ou du sujet du devoir.

\IDkey{sector} un secteur, une section, au sens administratif, à laquelle s'adresse le devoir.

\IDkey{class} la classe concernée par le devoir.

\IDkey{location} le lieu géographique où a lieu le devoir.

\IDkey{date} le texte donnant la date du devoir.

\IDkey{time} le texte indiquant juste la durée du devoir.

\end{document}
